\documentclass{article}
\pagenumbering{roman}

\title{ System-on-Chip Approach to High Performance
	Audio Signal Processing and Transport }

\author{ Ilya Dmitrichenko }

\begin{document}

\maketitle

\begin{abstract}
\em

 Looking at recent innovative products for general and
 application-specific embedded computing, majority is
 Systems-on-Chip, some integrate a greater deal of peripherals
 and most are low-coast and low-power; not to mention rapid
 improvements of various performance factors.
  However it turns out that in audio application area there
 are fewer innovative dedvices and most of the vendors in the
 field apply proproetary technologies. On large, these are
 featuring an FPGA along with system control computer and
 separate peripheral controllers.
 % of more general kind.

  Most of interesting SoC devices on the market, dominated by
 ARM cores, rather rearly include high performance FPU. Some
 icorporate GPUs. There are ways of using GPU for many types
 of signal processing applications, however not many had yet
 been found particularly versetile for audio. This may be an
 interesting subject, though the purpose of this project is
 to bring yet another platrform, the primary deployment area
 of which has somewhat different nature of requirements.
  For flexibility and other architectural reasons, porpular
 x86 family is not being considered. Although multiple other
 alternatives are to be studied and discussed.

\end{abstract}

%% that's including NetLogic MIPS64, ADI SHARC, TI OMAP,
%% XMOS XS1, TILE64, ..

  This project aims at developing a suitable hardware closely coupled
  with high-level software framework to facilitate research in the field
  of distributed audio processing and control, also capable to incorporate 
  real-time network streaming and data storage, applying latest technology
  standards.
  During the primary phase the main interest is to implement a platform
  using open-source software and hardware, with major orientation into
  the second, utilsing OpenSPARC architecture.
  OpenSPARC provides multi-core general-porpouse high performance host
  processor and by being open-source it can be greatly extended to provide 
  a configurable datapath pipeline with semi-fixed audio function units.
  It also features network and storage iterface controllers, therefore in
  later phase technologies such as Audio Video Bridging can be integrated
  to build a distributed system with multiple nodes of different function
  classes.

  Additional issues to be addressed in the project to certain extend,
  would be in the area of multi-threaded and multi-application host
  audio processes. Hardware framework would provide facilities to
  accellerate signal processing elements, which traditionally had
  been handled by software.  
  The configurable datapath with semi-fixed function blocks is to
  represent ... 


\end{document}
