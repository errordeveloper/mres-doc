\documentclass[twocolumn]{article}
\usepackage[utf8]{inputenc}
\usepackage{hyperref}
\pagenumbering{roman}

\title{ System-on-Chip Approach to High Performance
	Audio Signal Processing and Transport \\
	\emph{\small{MRes System Engineering Project Proposal (Draft)}} }

\author{ Ilya Dmitrichenko }


\begin{document}

\maketitle

\begin{abstract}
\em

 Looking at recent innovative products for general and
 application-specific embedded computing, majority is
 Systems-on-Chip, some integrate a greater deal of peripherals
 and most are low-coast and low-power; not to mention rapid
 improvements of various performance factors.
  However it appears that in audio application area there
 are fewer innovative devices and most of the vendors in the
 field apply proprietary technologies. On large, these are
 featuring an FPGA along with system control computer and
 separate peripheral controllers.
 % of more general kind.

\end{abstract}

\section{Introduction}

  Most of interesting SoC devices on the market, dominated by
 ARM cores, rather rarely include high performance FPU. Some
 incorporate GPUs. There are ways of using GPU for many types
 of signal processing applications, however not many had yet
 been found particularly versatile for audio. This may be an
 interesting subject, though the purpose of this project is
 to bring yet another platform, which had been primarily
 deployed in area with a different nature of requirements.
  For flexibility and other architectural reasons, popular
 x86 family is not being considered. Although multiple other
 alternatives are to be studied and discussed.


%% that's including NetLogic MIPS64, ADI SHARC, TI OMAP,
%% XMOS XS1, TILE64, ..

\section{Aims \& Objectives}

  This project aims at developing a suitable hardware closely coupled
  with high-level software framework to facilitate research in the field
  of distributed audio processing and control, also capable to incorporate 
  real-time network streaming and data storage, applying latest technology
  standards.
  During the primary phase main interest is to implement a platform
  using open-source software and hardware, with major orientation into
  the second, utilising OpenSPARC architecture \cite{weaver2008opensparc,
  page:opensparc:docs}.
  OpenSPARC provides multi-core general-purpose high performance host
  processor and by being open-source it can be greatly extended to provide 
  a configurable datapath pipeline with semi-fixed audio function units.
  It also features network and storage interface controllers, therefore in
  later phase technologies such as Audio Video Bridging \cite{page:avb:home,
  paper:avb:pro, paper:avb:intro} can be integrated to build a distributed
  system with multiple nodes of different function classes. 
  
  Various proprietary technologies exist, which provide accelerated
  audio processing. However, these are not accessible to researchers,
  only end-user application software is provided by most vendors.

%% Avid Pro-tools, Universal Audio ..

  Dedicated DSP chips can be utilised, however the algorithm code
  requires extensive modifications to run functions that DSP chip
  provides. Popular open-source software that is available to audio
  researchers \cite{wiki:pd,wiki:sc,wiki:cs} has been utilising host
  CPU (largely x86) to carry out most of DSP routines. This is a
  practically proven concept, though often when multichannel sound
  is concerned, it is desired that each channel is processed by equaliser
  and compressor before it is mixed. Another important aspect is
  floating-point conversion. These components often become particularly
  intensive in computation. It is not disagreed that GPU approach may
  be applied \cite{moore2009the,savioja2010real,tsingos2009using},
  though the aim is to build a single-chip device.
  The benefits of OpenSPARC are its free open-source license as well
  as advanced 64-bit architecture \cite{weaver1994sparc} with all
  necessary peripherals.

  Additional issues to be addressed in the project to certain extend,
  would be in the area of multi-threaded and multi-application host
  audio processing. Hardware embedded in the CPU chip would provide
  facilities to accelerate signal processing, which traditionally
  had been handled by software or analogue outboard. The configurable
  datapath with semi-fixed function blocks is modeling the analogue
  mixing console with outboard dynamics and equalisers, as well as
  patch-bays.

  This should be achieved without creating a new instruction set,
  i.e. only adding special function registers and a clock domain.
  Certainly, the memory controller will need to be coupled with
  a domain specific extension, providing an optimal software
  interface. With this architecture it should be possible to 
  incorporate such implementation into a different system,
  such as less expensive ARM-based solution or other.
  However, in the duration of first phase of the project, minimum
  complexity has to be taken care of. This implies that some of
  more advanced DSP units may have to be implemented in the later
  phase. The aim is to produce the basis for further work. 
  
  The purpose is dual, it is a very challenging research project
  as well as potential building block for commercial audio systems.

\section{Technical Requirements}

  OpenSPARC is a large multi-core SoC, however it is possible
  to build a smaller system \cite{page:opensparc:fpga} on FPGA,
  optimised for the size of available target device.
  

\bibliographystyle{plain}

\bibliography{../bibliography/books,../bibliography/papers,../bibliography/links}

\end{document}
